\documentclass[titlepage,12pt,a4paper]{article}

\usepackage[left=2cm,top=3cm,right=2cm,bottom=3cm,bindingoffset=0.5cm]{geometry}
\usepackage{amsmath}
\usepackage{amssymb}
\usepackage{enumitem}
\usepackage{commath}
\usepackage{mathtools}
\usepackage{graphicx}
\usepackage{dirtytalk}
\usepackage{csquotes}
\usepackage{hyperref}
\usepackage{tabto}
\usepackage{gensymb}
\usepackage{graphicx}
\usepackage{listings}
\usepackage{sidecap}
\usepackage{wrapfig}
\usepackage{parskip}

\usepackage{fancyhdr}
\usepackage{color}
\definecolor{dkgreen}{rgb}{0,0.6,0}
\definecolor{gray}{rgb}{0.5,0.5,0.5}
\definecolor{mauve}{rgb}{0.58,0,0.82}

\lstset{frame=tb,
  language=C++,
  aboveskip=3mm,
  belowskip=3mm,
  showstringspaces=false,
  columns=flexible,
  basicstyle={\small\ttfamily},
  numbers=left,
  numberstyle=\footnotesize,
  stepnumber=1,
  numbersep=5pt,
  keywordstyle=\color{blue},
  commentstyle=\color{dkgreen},
  stringstyle=\color{mauve},
  breaklines=true,
  breakatwhitespace=true,
  tabsize=3
}


\pagestyle{fancy}\lhead{A} \rhead{C}
\chead{{\large{\bf B}}}
\lfoot{}
\rfoot{\bf \thepage}
\cfoot{}

\setlength{\headheight}{15.2pt}
\pagestyle{fancy}
\fancyhf{}
\lhead{ \fancyplain{}{COMP3431: Robotic Software Architecture} }
\rfoot{ \fancyplain{}{\thepage} }


\begin{document}
\begin{titlepage}
    \begin{center}
        \vspace*{3cm}
        
        \Huge
        \textbf{COMP3431\\}
        \title{}
        \vspace{0.5cm}
        \Huge
        \textbf{Robotic Software Architecture}
        
        \vspace{0.54cm}
        
        \Large
        Assignment 2: Report
        
        \vspace{5cm}

	\large
	Nathan ADLER\\
	Aneita YANG\\

	\vfill
        
        \Large
        November 9, 2015
        
    \end{center}
\end{titlepage}

\pagebreak
\tableofcontents

\pagebreak
\section{Introduction}
In this assignment, both the hardware and software aspects of robotics were explored. The overall objective was to create a robot that would drive autonomously in an outdoor environment, whilst avoiding any obstacles. A motorised wheelchair was the baseline from which the robot was constructed.

To achieve the objective, the robot is equipped with a GPS and compass (using an Android phone with ROS). A laser scanner is also attached to the front of the robot, gathering information about the robot's immediate surroundings.

\subsection{Modules}
Five modules run in conjunction to operate the robot.

\pagebreak
\section{Hardware}

\pagebreak
\section{Software}

\pagebreak
\section{Results}

\pagebreak
\section{Future Work and Improvements}

\pagebreak
\section{Appendix}

\end{document}
